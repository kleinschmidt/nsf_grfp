\documentclass[12pt]{article}

\usepackage[margin=1in]{geometry}
\usepackage{fontspec}
\setmainfont[Mapping=tex-text]{Times New Roman}

%\usepackage{apacite}
\usepackage{cite}



\begin{document}
%A POSSIBLE OUTLINE:
%I. Why I think language is interesting

I've always had an interest in what I now recognize as cognitive science, but I developed a fascination with language and computational modeling as an undergraduate.  Introductory courses in cognitve science and philosophy introduced me to the big, foundational issues and ways of thinking that characterize modern cognitive science, and courses in phonetics and phonology introduced me to the deeply challenging and interesting problem of speech sounds, as well as a set of theoretic tools for describing and explaining them.  The problem of finding a satisfactory explanation of speech perception and production, in particular, and of intelligent behavior, in general, is particularly compelling to me for two reasons.  First, there are few other fields where collaboration between different disciplines is so potentially beneficial, and even necessary.  Second, there are similarly few fields where basic research [[[---> applications, benefits, etc.]]]

%II. Williams

My experience as an undergraduate at Williams College has prepared me to actively contribute to the field of cognitive science, both through my own work and by encouraging and enabling the work of others.  On the first count, the small and distinguished faculty at Williams [[[blah blah blah]]]

%  A. Opportunity to work closely with faculty and other students (mentoring students in Safa's lab) --- leadership roles

I consider myself incredibly fortunate to have had the opportunities I did at Williams to work closely with faculty and my fellow students, and as a result I believe strongly in the importance of mentoring students at the graduate and undergraduate level throughout my career.  As a member of Safa Zaki's lab, I actively mentored other undergraduate RAs with less research experience, assisting with the design, implementation, and analysis of a number of experiments, as well as sharing my mathematical expertise by answering questions and discussing the mathematical foundations of theories and analytic techniques.

%    1. a capella here?
%%% THIS NEEDS TO BE MORE CONCISE, FOCUSED

I took on leadership roles in other areas, most noteably as the director of a student-run a cappella music group for two years.  This was a serious commitment of time and energy, and involved not only leading rehearsals and conducting.  We were a somewhat rag-tag group, generally accepting members with a wider range of backgrounds and abilities than other similar groups, and on top of that learning an entirely new reproitoire every semester.  Also unlike many groups, we chose our reproitoire democratically.  While such diversity and democracy made leading effectively a real challenge, it also gave our group a vitality and liveliness that I strove to preserve, even as I pushed our development into a more musically talented and serious group.  This experience was humbling and extremely rewarding, and while the relevance of this experience to a career as a research scientist may not be immediately apparent, cognitive science research also requires getting people from different backgrounds to work together, often with very different ideas about what the goals of the enterprise are.  One cannot simply dictate and command but must negotiate and explain.

%  B. Opportunity to take a wide range of classes AND focus in-depth on cog sci --- broad perspective on my chosen field
%III. Study abroad

One of the things I found most challenging and rewarding about Williams was the opportunity it provided to study abroad in a Tibetan exile community in India.  The program focused primarily on anthropology and politics of the exile community, and in addition we studied Buddhist philosophy from ordained practitioners in a local monastery.  Through the study of the complicated relationship between philosophy, culture, and political power, I came to appreciate the importance of effective presentation of one's beliefs in as wide a domain as possible.

My decision to study abroad was the culmination of a persistent interest in making an honest attempt to understand the basic questions and concepts of other individuals and groups.  While this impulse is most familiar as an anthropological one---where it is applied to understanding differences between often highly distinct cultural groups---it is also, I believe, one that is indispensible for scientists, and ignored at the peril most especially of scientists in highly interdisciplinary fields such as cognitive science.  Truly collaborative interdisciplinary research requires that researchers from different fields temporarily put their own theoretical frameworks aside in order to understand and benefit from the theoretical frameworks, methods, and ideas of other fields.  


%  C. Relationship between science, philosophy, and culture
%IV. UMD

My time as a Baggett Fellow in the UMD Linguistics department was absolutely instrumental in focusing my research interests and reinforcing my desire to pursue a career as a researcher.  Participating in the day-to-day life of an active research university was incredibly stimulating, and I found that being able to contribute to this community very rewarding.  

One aspect that I found particularly rewarding was the language science IGERT program.  Along with IGERT students from linguistics, biology, and electrical engineering, I worked on applying some of the machine learning tools Bill Idsardi and I were investigating as models of phonetic category learning to representations of speech by single neurons in the auditory cortex of ferrets (a project which is still in its early stages).  I also gave a tutorial on ANOVA and mixed-effects modeling during the student-organized Winter Storm workshops.  In both situations, I found that I learned at least as much as I taught, both through preparation of materials and through discussions with others
.
%  A. Relationships w/ linguists (opposite of my own viewpoint...hard to get along but we both saw the other's side)
%  B. IGERT --- role of graduate training in bringing together people in different fields/departments
%    1. ferret project
%    2. winter storm stats
%V. Rochester
%  A. Extremely energized by collaborative environment --- particularly sharing between students.

Finally, the first few months of my graduate training at Rochester have been unbelievably stimulating, again largely due to the vibrancy and open-mindedness of the students and faculty.  

\end{document}