\documentclass[12pt]{article}

\usepackage[margin=1in]{geometry}
\usepackage{fontspec}
\setmainfont[Mapping=tex-text]{Times New Roman}

%\usepackage{apacite}
\usepackage{cite}



\begin{document}

% NEW ORGANIZATIONAL STRUCTURE:


% I. educational experience: broad, intimate, intense

%%% THIS PROBABLY NEEDS TO BE TRIMMED SUBSTANTIALLY
I first became explicitly aware of and interested in cognitive science when I visited Williams College as a prospective student, and sat in on a cognitive science seminar.  The professor led a lively, two-hour discussion on how the particular representational and philosophical problems posed by language fit into broader debates in cognitive science.  When I enrolled in this class myself, I found the material challenging and incredibly exciting.  Nurtured by the small but dedicated group of Williams Cognitive Science students and faculty---a computer scientist, a philosopher, a biologist, and two psychologists---I explored foundational issues in cognitive science and developed practical knowledge of mathematics, statistics, computer science, linguistics, and cognitive psychology.  My education in philosophy (largely at the hands of Profs. Joseph Cruz and Georges Dreyfus) provided me with an awareness of the subtlety and difficulty of the philosophical issues raised by cognitive science, as well as an ability to apprehend and appreciate the larger context and implications of specific empirical and theoretical work.  Classes in phonetics and phonology introduced me to a wealth of fascinating and challenging data about the sound-systems of human languages, and piqued my interest in the general problems posed by language and its representational systems.  Majoring in mathematics, I learned the central importance of finding the right way of representing a problem, an inescapable conclusion after years of writing proofs, and constantly being stuck on the first step, waiting for inspiration to strike.  This knowledge, along with an appreciation for the broader context and implications of representational choices, informs my thinking about cognitive science in a fundamental way, and is at the core of my commitment to the critical examination of basic representational choices that constitutes truly interdisciplinary research. 

Speech perception is interesting to me precisely because it is relevant to so many different fields, but not adequately explained by any of them alone.  It lies at the nexus of many of the foundational issues of modern cognitive science, including the influence of top-down versus bottom-up information, learning versus innately specified representations, domain-generality, and the challenge of integrating neuroscience with traditional behavioral and modeling techniques, to name just a few.  Moreover, and equally importantly, understanding speech and language perception has major practical significance, including the development of better automatic speech recognition systems and improved cochlear implants.

%%% WEAVE THIS IN BETTER, AND CONDENSE.
At Williams, I also had the opportunity to partcipate in a study abroad program in India with a Tibetan exile community, an experience that was challenging and rewarding in a way that complemented my course work.  The program focused primarily on anthropology and politics of the exile community.  In addition, we studied the Tibetan language, and Tibetan Buddhist philosophy with ordained practitioners from a local monastery.  This experience, which emphasized the complicated relationship between philosophy, culture, and political power, I came to appreciate the importance of effective presentation of one's beliefs in as wide a domain as possible.

My decision to study abroad was part of a persistent and continuing interest in making an honest attempt to understand the basic concepts and questions that other groups and individuals use to understand the world.  While this impulse is most familiar as an anthropological one---where it is applied to understanding differences between often dramatically different cultural groups---it is also, I believe, one that is indispensible for scientists, especially for those in interdisciplinary fields such as cognitive science.  Truly collaborative interdisciplinary research requires that researchers from different fields temporarily put their own theoretical frameworks aside in order to understand and benefit from the theoretical frameworks, methods, and ideas of other fields.  

% II. mentoring/leadership/collaboration

%%%%% 
At Williams, I learned the importance both of mentoring and of good working relationships with peers.  
%%%%% FIX ME!!!
Four of my classes (marked with a T on my transcript) were conducted as two-person tutorials, where each week one of the pair read a selection of papers on a single topic, prepared a short paper (or mastered a more difficult proof, in the case of my Mathematics tutorials), and presented and discussed it with the professor and other student.  
Many of my other classes were taught as small, discussion-focused seminars, where I learned to balance expressing my ideas confidently and giving close and honest consideration to the ideas of others.  
 Working in Prof. Zaki's lab, I benefitted both from working closely with her and from the opportunity to collaborate with and mentor other members of the lab.  

I took on leadership roles outside the classroom and lab as well, most notably serving as the musical director of Good Question, a student-run a cappella group.  This was a serious commitment of time and energy, and involved leading rehearsals and providing leadership and direction to a group with a great diversity of backgrounds.  I had to draw on my moderate amount of musical experience to help bring our less experienced members up to speed, while also helping the group learn an entirely new, democratically selected repertoire each semester and giving high-level musical direction.  The experience was exhausting, humbling, and deeply rewarding, and from it I learned how to lead a diverse group towards a common goal, a skill that I believe has and will serve me well in my career as an interdisciplinary researcher.



% III. participation in vibrant research community (UMD/Rochester)

After graduating from Williams, I spent a year at the University of Maryland Linguistics Department as a Baggett Fellow, which focused and intensified my desire to do cognitive science research.  I found the day-to-day life of a vibrant, collaborative, and interdisciplinary community of students and faculty to be incredibly energizing.  The conversations I had with both faculty and students helped to focus my ideas and place them in a broader context, both for specific research projects and for general theoretical commitments.  While I often disagreed with people on larger theoretical issues, we always treated each other and each other's ideas with respect and consideration.  

I particularly enjoyed being able to share my enthusiasm for and knowledge about statistical methods, computational models, and their mathematical underpinnings.  I gave a tutorial on data analysis with ANOVA and mixed-effect models during the student-organized Winter Storm workshops (slides and example code are available from my website), and I presented and explained work by Naomi Feldman (who has since joined the faculty there) on hierarchical Bayesian modeling of unsupervised lexical and phonetic category learning.

My current department has been even more stimulating and participatory.  I've been actively involved in two lab groups (those of Profs. Florian Jaeger---my primary advisor---and Alex Pouget), where I've presented my own work as well as other topics of interest.  In addition, I've presented multiple times in an unofficial seminar on eye-tracking methods for psycholinguistics, and participate in the meetings of Profs. Richard Aslin/Elissa Newport's lab group as well.

% IV.

In conclusion, my educational background, with its combination of breadth and depth, has prepared me extremely well to be an independent, creative, competent, and flexible cognitive scientist.  I am passionate about cognitive science, both for its intellectual challenges as well as its broad impact and potential to do real good in the world.  I am equally passionate about teaching and mentoring, and I recognize that by sharing my particular knowledge, background, and skills I can not only help others reach their full potential but can improve the quality of my own thinking and research.  In the classroom, lab, and elsewhere I have demonstrated that I have the ability to lead and collaborate effectively.  The quality of the teaching and mentoring I have received, as an undergraduate at Williams, Baggett Fellow at Maryland, and now doctoral candidate at Rochester have made a deep, lasting, and humbling impression on me.  Receipt of the NSF Graduate Research Fellowship will assist me in attaining my educational and research goals at the University of Rochester.



%%%%%%%%%%%%%%%%%%
% SOME THINGS
%
% Where is TAing going to fit in? (second section)


\end{document}





