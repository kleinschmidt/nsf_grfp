\documentclass[12pt]{article}

\usepackage[margin=1in]{geometry}
\usepackage{fontspec}
\setmainfont[Mapping=tex-text]{Times New Roman}

%\usepackage{apacite}
\usepackage{cite}



\begin{document}

% NEW ORGANIZATIONAL STRUCTURE:


% I. educational experience: broad, intimate, intense

%%% THIS PROBABLY NEEDS TO BE TRIMMED SUBSTANTIALLY
I first became explicitly aware of and interested in cognitive science when I was visiting Williams College as a prospective student, and sat in on a cognitive science seminar, where the professor lead a lively, two-hour discussion on how the particular representational and philosophical problems posed by language fit into broader debates in cognitive science.  When I enrolled in this class myself (the very next spring), I found the material challenging and incredibly exciting.  Nurtured by the small but dedicated group of Williams Cognitive Science students and faculty---one computer scientist, one philosopher, one biologist, and two psychologists---I explored foundational issues in cognitive science and developed practical knowledge of mathematics, statistics, computer science, linguistics, and cognitive psychology.  My education in philosophy (largely at the hands of Joe Cruz and Georges Dreyfus) endowed me with an awareness of the subtlety and difficulty of the philosophical issues raised by cognitive science, as well as an ability to apprehend and appreciate the larger context and implications of specific empirical and theoretical work.  Classes in phonetics and phonology introduced me to a wealth of fascinating and challenging data about the sound-systems of human languages, and really piqued my interest in the general problems posed by language and its representational systems.  Majoring in mathematics, I learned the central importance of finding the right way of representing a problem, an inescapable conclusion after years of writing proofs, and constantly being stuck on the first step, waiting for inspiration to strike.  This knowledge, along with an appreciation for the broader context and implications of representational choices, informs my thinking about cognitive science in a fundamental way, and is at the core of my commitment to the critical examination of basic representational choices that constitutes truly interdisciplinary research. 

Speech perception is interesting to me precisely because it is relevant to so many different fields, but not adequately explained by any of them alone.  It lies at the nexus of many of the foundational issues of modern cognitive science, including the influence of top-down versus bottom-up information, learning versus innately specified representations, domain-generality, and the challenge of integrating neuroscience with traditional behavioral and modeling techniques, to name a few.  Moreover, and equally importantly, understanding speech and language perception has major practical significance, including the development of better automatic speech recognition systems and improved cochlear implants.

%%% WEAVE THIS IN BETTER, AND CONDENSE.
One of the things I found most challenging and rewarding about Williams was the opportunity it provided to study abroad in a Tibetan exile community in India.  The program focused primarily on anthropology and politics of the exile community, and in addition we studied Buddhist philosophy from ordained practitioners in a local monastery.  Through the study of the complicated relationship between philosophy, culture, and political power, I came to appreciate the importance of effective presentation of one's beliefs in as wide a domain as possible.

My decision to study abroad was the culmination of a persistent interest in making an honest attempt to understand the basic questions and concepts of other individuals and groups.  While this impulse is most familiar as an anthropological one---where it is applied to understanding differences between highly distinct cultural groups---it is also, I believe, one that is indispensible for scientists, especially for those in interdisciplinary fields such as cognitive science.  Truly collaborative interdisciplinary research requires that researchers from different fields temporarily put their own theoretical frameworks aside in order to understand and benefit from the theoretical frameworks, methods, and ideas of other fields.  

% II. mentoring/leadership/collaboration

At Williams, I learned the importance both of mentoring and of good working relationships with peers.  Four of my classes (marked with a T on my transcript) were conducted as two-person tutorials, where each week one of the pair read a selection of papers on a single topic, prepared a short paper (or mastered a more difficult proof, in the case of my Mathematics tutorials), and presented and discussed it with the professor and other student.  
Many of my other classes were taught as small, discussion-focused seminars, where I learned the difficult balance between expressing my ideas confidently while given close and honest consideration to the ideas of others.  
 Working in Dr. Zaki's lab, I benefitted both from my close relationship with Dr. Zaki but just as much from the opportunity to collaborate with and mentor other members of the lab.  

I took on leadership roles outside the classroom and lab as well, most notably serving as the musical director of Good Question, a student-run a cappella group.  This was a serious commitment of time and energy, and involved not only leading rehearsals but providing leadership and direction to a group with a great diversity of backgrounds.  I had to draw on my moderate amount of musical experience to help bring our less experienced members up to speed, while also helping the group learn an entirely new, democratically selected repertoire each semester and giving high-level musical direction.  The experience was exhausting, humbling, and deeply rewarding, and from it I learned how to lead a diverse group towards a common goal, a skill that I believe has and will serve me well in my career as an interdisciplinary researcher.



% III. participation in vibrant research community (UMD/Rochester)

The year that I spent at the University of Maryland Linguistics Department as a Baggett Fellow focused and intensified my desire to do cognitive science research.  I found the day-to-day life of a vibrant, collaborative, and interdisciplinary community of students and faculty to be incredibly energizing.  The conversations I had with both faculty and students helped to focus and my ideas and place them in a broader context, both for specific research projects and for general theoretical commitments.  While I often disagreed with people on larger theoretical issues, we always treated each other with respect and consideration.  

I particularly enjoyed being able to share my enthusiasm for and knowledge about statistical methods, computational models, and their mathematical underpinnings.  I gave a tutorial on data analysis with ANOVA and mixed-effect models during the student-organized Winter Storm workshops (slides and example code are available from my website), and I presented and explained work by Naomi Feldman (who has since joined the faculty there) on hierarchical Bayesian modeling of unsupervised lexical and phonetic category learning.

I have found my current department to be even more stimulating and participatory.  So far, I've been actively involved in two lab groups (those of my primary advisor, Florian Jaeger, and Alex Pouget), where I've presented my own work and others' that I thought would be interesting to the lab.  In addition, I've presented multiple times in an unofficial seminar on eye-tracking methods for psycholinguistics, and participate in the Richard Aslin/Elissa Newport lab group's meetings as well.

% IV.

In conclusion, my educational background, with its combination of breadth and depth, has prepared me extremely well to be an independent, creative, competent, and flexible cognitive scientist.  I am passionate about cognitive science, both for its intellectual challenges as well as its broad impact and potential to do real good in the world.  I am equally passionate about teaching and mentoring, and I recognize that by sharing my particular knowledge, background, and skills I can not only help others reach their full potential but can improve the quality of my own thinking and research.  In the classroom, lab, and elsewhere I have demonstrated that I have the ability to lead and collaborate effectively.  The quality of the teaching and mentoring I have received, both as an undergraduate at Williams and as a graduate student and Baggett Fellow at Maryland, have made a deep, lasting, and humbling impression on me.  

%My time as a Baggett Fellow in the UMD Linguistics department was absolutely instrumental in focusing my research interests and reinforcing my desire to pursue a career as a researcher.  Participating in the day-to-day life of an active research university was incredibly stimulating, and I found that being able to contribute to this community very rewarding.  

%One aspect that I found particularly rewarding was the language science IGERT program.  Along with IGERT students from linguistics, biology, and electrical engineering, I worked on applying some of the machine learning tools Bill Idsardi and I were investigating as models of phonetic category learning to representations of speech by single neurons in the auditory cortex of ferrets (a project which is still in its early stages).  I also gave a tutorial on ANOVA and mixed-effects modeling during the student-organized Winter Storm workshops.  In both situations, I found that I learned at least as much as I taught, both through preparation of materials and through discussions with others

%Finally, the first few months of my graduate training at Rochester have been unbelievably stimulating, again largely due to the vibrancy and open-mindedness of the students and faculty.  

\end{document}





